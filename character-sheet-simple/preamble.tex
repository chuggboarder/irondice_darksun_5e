%%%%%%%% CREATE DOCUMENT STRUCTURE %%%%%%%%
%% Language and font encodings
\usepackage[english]{babel}
\usepackage[utf8]{inputenc}
\usepackage[T1]{fontenc}
\usepackage{ebgaramond}

%% Sets page size and margins
\usepackage[a4paper,top=2cm,bottom=2cm,left=2cm,right=2cm,marginparwidth=1.75cm]{geometry}

%% Useful packages
\usepackage{amsmath}
\usepackage{amsfonts}
\usepackage[colorlinks=true, allcolors=black]{hyperref}
\usepackage[font=small,labelfont=bf]{caption}
\usepackage{subcaption}
\usepackage{lipsum}
\usepackage{xcolor}
\usepackage{xurl}
\usepackage{csquotes}
\usepackage[toc,page]{appendix}
\usepackage{enumitem}
\usepackage{graphicx}
\usepackage{rotating}
\usepackage{wrapfig}
\usepackage{tikz}
\usepackage{titling}
\usepackage{float}
\usepackage{tabularx}
\usepackage{ltablex}
\usepackage{colortbl}
\usepackage{hhline}
\usepackage{multirow}
\usepackage{multicol}
\usepackage{watermark}
\usepackage{yfonts}
\usepackage{fdsymbol}

% bibliography settings
\usepackage[style=apa, backend=biber]{biblatex}
\bibliography{ref}

% Defining new editor type for Biblatex
\NewBibliographyString{illustrator}
\NewBibliographyString{illustrators}
\NewBibliographyString{byillustrator}
\NewBibliographyString{cbyillustrator}
\NewBibliographyString{typeillustrator}
\NewBibliographyString{typeillustrators}

\DefineBibliographyStrings{english}{%
  references       = {Works Used},
  illustrator      = {illustrator},
  illustrators     = {illustrators},
  typeillustrator  = {illustrator},
  typeillustrators = {illustrators},
  byillustrator    = {illustrated by},
  cbyillustrator   = {illustr\adddot},
}

% The important DnD red
\definecolor{maroon}{HTML}{4a2b18}

% It sets your hline colour
\arrayrulecolor{maroon}


% Change \texttt{} to split on a line like a url.
\renewcommand{\texttt}[1]{%
  \begingroup
  \ttfamily
  \begingroup\lccode`~=`/\lowercase{\endgroup\def~}{/\discretionary{}{}{}}%
  \begingroup\lccode`~=`[\lowercase{\endgroup\def~}{[\discretionary{}{}{}}%
  \begingroup\lccode`~=`.\lowercase{\endgroup\def~}{.\discretionary{}{}{}}%
  \catcode`/=\active\catcode`[=\active\catcode`.=\active
  \scantokens{#1\noexpand}%
  \endgroup
}

% color large sections by encapsulating them into a single command: \colorsections{color}{text}.
\newcommand{\colorsections}[2]{{\leavevmode\color{#1}#2}}

% circles text
\newcommand*\circled[1]{\tikz[baseline=(char.base)]{
            \node[shape=circle,draw,inner sep=1pt,line width=0.2pt] (char) {#1};}}

% Skull pictogram command
\DeclareFontFamily{U}{skulls}{}
\DeclareFontShape{U}{skulls}{m}{n}{ <-> skull }{}
\newcommand{\skull}{\text{\usefont{U}{skulls}{m}{n}\symbol{'101}}}


%%% Format section headers
\makeatletter

\let\oldsection\section
\let\oldsubsection\subsection

\def\section{\@ifstar\s@section\@section}
\def\subsection{\@ifstar\s@subsection\@subsection}

% numbered section
\newcommand{\@section}[2][\relax]{
% #1 is the section counter; #2 is the section name.
% #1 hates being changed (colour is ok); #2 can be freely stylised.
    \vspace{-0.5em}
        % change format of section name here
        \oldsection[\textcolor{maroon}{#1}]{\textcolor{maroon}{\textsc{#2}}}
    \vspace{-0.2em}
    \hrule
    \vspace{0.7em}
}

% unnumbered section
\newcommand{\s@section}[1]{
% unnumbered (sub)sections do not have a counter,
% so #1 is the section name
    \vspace{-0.5em}
        \oldsection*{\textcolor{maroon}{\textsc{#1}}}
    \vspace{-0.2em}
    \hrule
    \medskip
}

% numbered subsection
\newcommand{\@subsection}[2][\relax]{
    \vspace{-0.5em}
        \oldsubsection[\textcolor{maroon}{#1}]{\textcolor{maroon}{\textsc{#2}}}
    \vspace{-0.2em}
}

% unnumbered subsection
\newcommand{\s@subsection}[1]{
    \vspace{-0.5em}
        \oldsubsection*{\textcolor{maroon}{\textsc{#1}}}
    \vspace{-0.2em}
}
\makeatother

% Not using the titlesec package for this due to this issue:
% https://github.com/jbezos/titlesec/issues/64
% But this is how you could:

% \usepackage{titlesec}
% \titlespacing*{\section}{0pt}{2mm}{1mm}
% \titlespacing*{\subsection}{0pt}{2mm}{1mm}
% \titleformat{\section}{\color{maroon}\Large\bfseries\scshape}{\thesection}{0.5em}{}[\titlerule\vspace*{4pt}]
% \titleformat{\subsection}{\color{maroon}\large\bfseries\scshape}{\thesubsection}{0.5em}{}


% Change footnote style to roman numerals and make the line maroon
\renewcommand{\thefootnote}{\Roman{footnote}}
\renewcommand{\footnoterule}{%
  \kern -3pt
  {\color{maroon}\hrule width 0.4\columnwidth height 0.5pt}
  \kern 2.6pt
}

% make a line
\newcommand{\HRule}{\color{maroon}\rule{\linewidth}{0.5mm}}
